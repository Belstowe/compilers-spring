\documentclass[]{article}
\usepackage[default]{opensans}
\usepackage[utf8]{inputenc}
\usepackage[T1,T2A]{fontenc}
\usepackage[russian]{babel}
\usepackage{amsmath}
\usepackage{color}
\usepackage{listings}
\usepackage{pxfonts}

\usepackage[margin=1mm]{geometry}
\usepackage[bottomtitles]{titlesec}
\widowpenalty=0\relax
\clubpenalty=0\relax
\pagestyle{empty}

\lstset{basicstyle=\rmfamily,
        breaklines=true,
        escapechar=!,
        columns=fullflexible,
        mathescape=true,
        literate=
               {=>}{$\Rightarrow{}$}{1}}

\begin{document}

%--------------------------------------------------------------------------------
\section{Формальные языки, грамматики и их свойства}

\begin{enumerate}

\begin{item}
    Дана грамматика. Постройте вывод заданной цепочки.
    \begin{enumerate}
    
    \begin{item}
        \begin{lstlisting}
            S : T | T '+' S | T '-' S;
            T : F | F '*' T;
            F : 'a' | 'b';
        \end{lstlisting}
        Цепочка: \lstinline|a - b * a + b|.
        \bigbreak
        \textbf{Решение:}
        \begin{lstlisting}
            !\textcolor{red}{S}! => !\textcolor{red}{T}! - S => F - !\textcolor{red}{S}! => !\textcolor{red}{F}! - T + S =>
                a - !\textcolor{red}{T}! + S => a - F * T + !\textcolor{red}{S}! => 
                    => a - !\textcolor{red}{F}! * !\textcolor{red}{T}! + !\textcolor{red}{T}! => a - b * !\textcolor{red}{F}! + !\textcolor{red}{F}! =>
                            => a - b * a + b
        \end{lstlisting}
    \end{item}
    
    \begin{item}
        \begin{lstlisting}
            S : 'a' S B C | 'ab' C;
            C B : B C;
            'b' B : 'bb';
            'b' C : 'bc';
            'c' C : 'cc';
        \end{lstlisting}
        Цепочка: \lstinline|aaabbbccc|.
        \bigbreak
        \textbf{Решение:}
        \begin{lstlisting}
            !\textcolor{red}{S}! => a !\textcolor{red}{S}! B C => aa !\textcolor{red}{S}! B C B C => aaab !\textcolor{red}{C B}! C B C
                => aaa!\textcolor{red}{b B}! C C B C => aaabb C !\textcolor{red}{C B}! C =>
                    => aaabb !\textcolor{red}{C B}! C C => aaab!\textcolor{red}{b B}! C C C =>
                        => aaabb!\textcolor{red}{b C}! C C => aaabbb!\textcolor{red}{c C}! C =>
                            => aaabbbc!\textcolor{red}{c C}! => aaabbbccc
        \end{lstlisting}
    \end{item}
    
    \end{enumerate}
\end{item}

\begin{item}
    Построить грамматику, порождающую язык:
    \begin{enumerate}
    
        \begin{item}
            $L = \{ a^n b^m c^k\ |\  n, m, k > 0 \}$
            \bigbreak
            \textbf{Решение:}
            \begin{lstlisting}
                G ({a, b, c}, {S, A, B, C}, P, S)
                P:
                    S => aA
                    A => aA | B
                    B => bB | bC
                    C => cC | c
            \end{lstlisting}
        \end{item}

        \begin{item}
            $L = \{ 0^n (10)^m \ |\  n, m >= 0 \}$
            \bigbreak
            \textbf{Решение:}
            \begin{lstlisting}
                G ({0, 1}, {S, A, B}, P, S)
                P:
                    S => 0S | 1A | !$\varepsilon$!
                    A => 0B | 0
                    B => 1A
            \end{lstlisting}
        \end{item}

        \begin{item}
            $L = \{ a_1 a_2 \dots a_n a_n \dots a_2 a_1 \ |\  a_i \in \{ 0, 1 \} \}$
            \bigbreak
            \textbf{Решение:}
            \begin{lstlisting}
                G ({0, 1}, {S, A, B}, P, S)
                P:
                    S => 0A | 1B
                    A => S0 | !$\varepsilon$!
                    B => S1 | !$\varepsilon$!
            \end{lstlisting}
        \end{item}
    
    \end{enumerate}
\end{item}

\end{enumerate}

%--------------------------------------------------------------------------------

\end{document}