\documentclass[]{article}
\usepackage[default]{opensans}
\usepackage[utf8]{inputenc}
\usepackage[T1,T2A]{fontenc}
\usepackage[russian]{babel}
\usepackage{amsmath}
\usepackage{color}
\usepackage{listings}
\usepackage{pxfonts}
\usepackage{hyperref}
\usepackage{graphicx}

\usepackage[margin=1mm]{geometry}
\usepackage[bottomtitles]{titlesec}
\widowpenalty=0\relax
\clubpenalty=0\relax
\pagestyle{empty}

\lstset{basicstyle=\rmfamily,
        breaklines=true,
        escapechar=!,
        columns=fullflexible,
        mathescape=true,
        literate=
               {=>}{$\Rightarrow{}$}{1}}

\begin{document}

%--------------------------------------------------------------------------------
\section{Формальные языки, грамматики и их свойства}

\begin{enumerate}

\begin{item}
    Дана грамматика. Постройте вывод заданной цепочки.
    \begin{enumerate}
    
    \begin{item}
        \begin{lstlisting}
            S => T | T '+' S | T '-' S;
            T => F | F '*' T;
            F => 'a' | 'b';
        \end{lstlisting}
        Цепочка: \lstinline|a - b * a + b|.
        \bigbreak
        \textbf{Решение:}
        \begin{lstlisting}
            !\textcolor{red}{S}! => !\textcolor{red}{T}! - S => F - !\textcolor{red}{S}! => !\textcolor{red}{F}! - T + S =>
                a - !\textcolor{red}{T}! + S => a - F * T + !\textcolor{red}{S}! => 
                    => a - !\textcolor{red}{F}! * !\textcolor{red}{T}! + !\textcolor{red}{T}! => a - b * !\textcolor{red}{F}! + !\textcolor{red}{F}! =>
                            => a - b * a + b
        \end{lstlisting}
    \end{item}
    
    \begin{item}
        \begin{lstlisting}
            S => 'a' S B C | 'ab' C;
            C B => B C;
            'b' B => 'bb';
            'b' C => 'bc';
            'c' C => 'cc';
        \end{lstlisting}
        Цепочка: \lstinline|aaabbbccc|.
        \bigbreak
        \textbf{Решение:}
        \begin{lstlisting}
            !\textcolor{red}{S}! => a !\textcolor{red}{S}! B C => aa !\textcolor{red}{S}! B C B C => aaab !\textcolor{red}{C B}! C B C
                => aaa!\textcolor{red}{b B}! C C B C => aaabb C !\textcolor{red}{C B}! C =>
                    => aaabb !\textcolor{red}{C B}! C C => aaab!\textcolor{red}{b B}! C C C =>
                        => aaabb!\textcolor{red}{b C}! C C => aaabbb!\textcolor{red}{c C}! C =>
                            => aaabbbc!\textcolor{red}{c C}! => aaabbbccc
        \end{lstlisting}
    \end{item}
    
    \end{enumerate}
\end{item}

\begin{item}
    Построить грамматику, порождающую язык:
    \begin{enumerate}
    
        \begin{item}
            $L = \{ a^n b^m c^k\ |\  n, m, k > 0 \}$
            \bigbreak
            \textbf{Ответ:}
            \begin{lstlisting}
                G ({a, b, c}, {S, A, B, C}, P, S);
                P:
                    S => 'a' A;
                    A => 'a' A | B;
                    B => 'b' B | 'b' C;
                    C => 'c' C | 'c';
            \end{lstlisting}
        \end{item}

        \begin{item}
            $L = \{ 0^n (10)^m \ |\  n, m >= 0 \}$
            \bigbreak
            \textbf{Ответ:}
            \begin{lstlisting}
                G ({0, 1}, {S, A, B}, P, S);
                P:
                    S => '0' S | '1' A | !$\varepsilon$!;
                    A => '0' B | 0;
                    B => '1' A;
            \end{lstlisting}
        \end{item}

        \begin{item}
            $L = \{ a_1 a_2 \dots a_n a_n \dots a_2 a_1 \ |\  a_i \in \{ 0, 1 \} \}$
            \bigbreak
            \textbf{Ответ:}
            \begin{lstlisting}
                G ({0, 1}, {S, A, B}, P, S);
                P:
                    S => '0' A | '1' B;
                    A => S '0' | !$\varepsilon$!;
                    B => S '1' | !$\varepsilon$!;
            \end{lstlisting}
        \end{item}
    
    \end{enumerate}
\end{item}

\begin{item}
    К какому типу по Хомскому относится грамматика с приведенными правилами? Аргументируйте ответ.
    \begin{enumerate}
        
        \begin{item}
            \begin{lstlisting}
                S => '0' A '1' | '01';
                '0' A => '00' A '1';
                A => '01';
            \end{lstlisting}
            \bigbreak
            \textbf{Решение:}
            \smallbreak
            КЗ грамматика:
            \hfill\break В левой и правой частях выражений не более 1 нетерминала $\Rightarrow$ грамматика соответствует 1 типу.
            \smallbreak
            КС грамматика:
            \hfill\break \lstinline|'0' A => '00' A '1'|: В левой части присутствуют терминальные символы.
            \bigbreak
            \textbf{Ответ:}
            Контекстно-зависимая грамматика (1 тип).
        \end{item}

        \begin{item}
            \begin{lstlisting}
                S => A 'b';
                A => A 'a' | 'ba';
            \end{lstlisting}
            \bigbreak
            \textbf{Решение:}
            \smallbreak
            КЗ грамматика:
            \hfill\breakВ левой и правой частях выражений не более 1 нетерминала $\Rightarrow$ грамматика соответствует 1 типу.
            \smallbreak
            КС грамматика:
            \hfill\breakВ левых частях выражений отсутствуют терминалы $\Rightarrow$ грамматика соответствует 2 типу.
            \smallbreak
            Регулярная грамматика:
            \hfill\breakЯвляется \emph{леволинейной} грамматикой, т.к. нетерминалы в правой части всегда следуют перед терминалами.
            \bigbreak
            \textbf{Ответ:}
            Регулярная грамматика (3 тип).
        \end{item}

    \end{enumerate}
\end{item}

\begin{item}
    Построить КС-грамматику, эквивалентную грамматике с правилами:
    \begin{lstlisting}
        S => A B | A B S;
        A B => B A;
        B A => A B;
        A => 'a';
        B => 'b';
    \end{lstlisting}

    \textbf{Решение:}
    \smallbreak
    $L \ni \{ ab; ba; abab; abba; baab; baba; \dots \};$
    \hfill\break$L = \{ a_1 a_2 \dots a_n \ |\  a_i \in \{ ab, ba \} \};$
    \bigbreak
    \textbf{Ответ:}
    \begin{lstlisting}
        G ({ab, ba}, {S, A}, P, S);
        P:
            S => A | A S;
            A => 'ab' | 'ba';
    \end{lstlisting}
\end{item}

\begin{item}
    Построить регулярную грамматику, эквивалентную грамматике с правилами:
    \begin{lstlisting}
        S => A '.' A;
        A => B | B A;
        B => '0' | '1';
    \end{lstlisting}

    \textbf{Решение:}
    \smallbreak
    $L \ni \{ 0.0; 0.1; 1.0; 1.1; 00.0; 0.00; 10.0; 11.0; \dots \};$
    \hfill\break$L = \{ a_1 a_2 \dots a_n . a_{n+1} a_{n+2} \dots a_{n+m} \ |\  a_i \in \{ 0, 1 \}; n, m > 0 \};$
    \bigbreak
    \textbf{Ответ:}
    \begin{lstlisting}
        G ({0, 1}, {S, A, B}, P, S);
        P:
            S => '0' A | '1' A;
            A => '0' A | '1' A | '.' B;
            B => '0' C | '1' C | '0' | '1';
    \end{lstlisting}
\end{item}

\begin{item}
    Напишите регулярное выражение для:
    \begin{enumerate}

        \begin{item}
            Множества идентификаторов, где идентификатор – это последовательность букв или цифр, начинающаяся с буквы или \_.
            \bigbreak
            \textbf{Ответ:}
            \begin{enumerate}

                \begin{item}
                    \begin{lstlisting}
                        (?<={)((([A-Za-z][A-Za-z0-9]*)|(_[A-Za-z0-9]+))((?=})|;))+(?=})
                    \end{lstlisting}
                \end{item}

                \begin{item}
                    \begin{itemize}
                        \item \lstinline|{Imposter420} $\Rightarrow$ Imposter420|
                        \item \lstinline|{_420ImPoster} $\Rightarrow$ _420ImPoster|
                        \item \lstinline|{__imposter} $\Rightarrow$ $\varepsilon$|
                        \item \lstinline|{420ImPoster} $\Rightarrow$ $\varepsilon$|
                        \item \lstinline|{Imp@ster} $\Rightarrow$ $\varepsilon$|
                        \item \lstinline|{Imposter420;_420ImPoster;coolGuy} $\Rightarrow$ Imposter420;_420ImPoster;coolGuy|
                        \item \lstinline|{_420ImPoster;__imp@sSt3r} $\Rightarrow$ $\varepsilon$|
                        \item \lstinline|{_} $\Rightarrow$ $\varepsilon$|
                        \item \lstinline|Imposter420 $\Rightarrow$ $\varepsilon$|
                    \end{itemize}
                \end{item}

                \item \url{https://regexr.com/6guc4}

            \end{enumerate}
        \end{item}

        \begin{item}
            Множества вещественных констант с плавающей точкой, состоящих из целой части, десятичной точки, дробной части, символа е или Е, целого показателя степени с необязательным знаком и необязательного суффикса типа – одной из букв f, F, l или L. Целая и дробная части состоят из последовательностей цифр. Может отсутствовать либо целая, либо дробная часть (но не обе сразу).
            \bigbreak
            \textbf{Ответ:}
            \begin{enumerate}

                \begin{item}
                    \begin{lstlisting}
                        (?<={)((((-|\+)?(0|[1-9][0-9]*)|\.[0-9]+|(-|\+)?(0|[1-9][0-9]*)\.[0-9]+)((e|E)(-|\+)?[0-9]+)?(f|F|l|L)?)((?=})|;))+(?=})
                    \end{lstlisting}
                \end{item}

                \begin{item}
                    \begin{itemize}
                        \item \lstinline|{5;7;-8}; $\Rightarrow$ 5;7;-8|
                        \item \lstinline|{0} $\Rightarrow$ 0|
                        \item \lstinline|0 $\Rightarrow$ $\varepsilon$|
                        \item \lstinline|{-25.6;.01;2} $\Rightarrow$ -25.6;.01;2|
                        \item \lstinline|{6;TheLetter;32} $\Rightarrow$ $\varepsilon$|
                        \item \lstinline|{6e+2;4.6E-1;-3.1e0}; $\Rightarrow$ 6e+2;4.6E-1;-3.1e0|
                        \item \lstinline|{6e} $\Rightarrow$ $\varepsilon$|
                        \item \lstinline|{4e+} $\Rightarrow$ $\varepsilon$|
                        \item \lstinline|{-.5} $\Rightarrow$ $\varepsilon$|
                        \item \lstinline|{5L;-32l;24.55f;-21F} $\Rightarrow$ 5L;-32l;24.55f;-21F|
                        \item \lstinline|{-0.216e+6F} $\Rightarrow$ -0.216e+6F|
                        \item \lstinline|{4e+5.4} $\Rightarrow$ $\varepsilon$|
                    \end{itemize}
                \end{item}

                \item \url{https://regexr.com/6gudb}

            \end{enumerate}
        \end{item}

    \end{enumerate}

\end{item}

\begin{item}
    Для регулярных выражений из предыдущего задания постройте конечные автоматы. Изобразите их в виде графа.
\end{item}

\end{enumerate}

%--------------------------------------------------------------------------------

\end{document}